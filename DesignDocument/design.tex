\documentclass[letterpaper,10pt,titlepage,draftclsnofoot,onecolumn,onesided] {IEEEtran}
\usepackage{listings}
\usepackage{underscore}
\usepackage[bookmarks=true]{hyperref}
\usepackage[utf8]{inputenc}
\usepackage[english]{babel}
\usepackage{titling}
\usepackage{graphicx}
\usepackage[noadjust]{cite}
\nocite{*}
\usepackage{abstract}

\hypersetup{
    bookmarks=false,    % show bookmarks bar?
    pdftitle={Technology Review},    % title
    pdfauthor={Cramer Smith, Sam Lichlyter, Eric Winkler, Zach Schneider},                     % author
    pdfsubject={Design Document},                        % subject of the document
    pdfkeywords={IFT, Design, Postal}, % list of keywords
    colorlinks=true,       % false: boxed links; true: colored links
    linkcolor=black,       % color of internal links
    citecolor=black,       % color of links to bibliography
    filecolor=black,        % color of file links
    urlcolor=blue,        % color of external links
    linktoc=page            % only page is linked
} 

% Document Title:
\def\doctitle{A Tool to Automatically Organize the Structure of a Codebase Using Information Foraging Theory Design Patterns}
\def\doctype{Design Document}
\def\team{Team Postal | Group \#38}

\markboth{Oregon State University}{\doctitle}

\begin{document}

\title{\Huge{\bfseries{\textsf{\doctitle}}}\\\textsf{\Large{\doctype}}\\\textsf{\large{\team}}}
\author{Cramer Smith, Sam Lichlyter, Eric Winkler, Zach Schneider}

\maketitle
\vfill
%\begin{abstract}
%\end{abstract}
\vfill

\pagebreak

\tableofcontents

\pagebreak

% 1
\section{Overview}

% 1.1 
\subsection{Scope}
This document will cover the design of the Postal Extension written for the Visual Studio Code integrated development environment. 
The focus of the design will be on the four main parts of the extension, and the use of Information Foraging Theory within the extension.
The four parts of the extension design are the parser, the data structure, the interface with Visual Studio Code and the user interface.
The document with go through each of these parts and describe in detail how each will be implemented and how each part will function.
The Information Foraging Theory Patters that are planned to be explored within the extension are the Specification Matcher, Structural Relatedness, Impact Location, Path Search, and Recollection.
The document will go into more detail as to what these patterns mean and how they will influence the design of the extension.

% 1.2
\subsection{Purpose}
This design document describes the planned design and steps for implementing the Postal extension for Visual Studio Code. 
The team implementing the design will use this document as the blueprint for the implementation of the extension. 

% 1.3
\subsection{Intended Audience}
This document is meant for the design stakeholders. 
That includes the team implementing the extension, their client, and their supervisors. 
Their supervisors being the people grading us on the implementation of the designs described within this document.

% 1.4 
\subsection{Conformance}
The document conforms to the IEEE Std 1016-2009.
% TODO

% 2
\section{Definitions}
% TODO

% 3
\section{Conceptual Model for Software Design Descriptions}
The software will be written with a model view control design pattern.
The main parts of the extension will fill these roles.
The model will be the data structure that we will use to represent the parsed files. 
The view will be the user interface and the integrated development environment.
The IDE and the user interface that the extension creates will be the view and act dependent of each other.
The control will be the event handlers that are 

% 3.1
\subsection{Software Design in Context}

% 3.2
\subsection{Software Design Descriptions Within the Life Cycle}
\subsubsection{Influences on SDD Preparation}
\subsubsection{Influences on Software Life Cycle Products}
\subsubsection{Design Verification and Design Role in Validation}

% 4
\section{Design Description Information Content}

% 4.1
\subsection{Introduction}

% 4.2
\subsection{SDD identification}

% 4.3
\subsection{Design stakeholders and their concerns}

% 4.4
\subsection{Design views}

% 4.5
\subsection{Design viewpoints}

% 4.5
\subsection{Design elements}
\subsubsection{Design Entities}
\subsubsection{Design Attributes}
\paragraph{Name Attribute}
\paragraph{Type Attribute}
\paragraph{Purpose Attribute}
\paragraph{Author Attribute}
\subsubsection{Design Relationships}
\subsubsection{Design Constraints}

% 4.6
\subsection{Design Overlays}

% 4.7
\subsection{Design Rationale}

% 4.8
\subsection{Design Languages}

% 5
\section{Design Viewpoints}

% 5.1
\subsection{Introduction}

% 5.2
\subsection{Context Viewpoint}
\subsubsection{Design Concerns}
\subsubsection{Design Elements}
\subsubsection{Example Languages}

% 5.3
\subsection{Composition Viewpoint}
\subsubsection{Design Concerns}
\subsubsection{Design Elements}
\subsubsection{Function Attribute}
\subsubsection{Example Languages}

% 5.4
\subsection{Logical Viewpoint}
\subsubsection{Design Concerns}
\subsubsection{Design Elements}

% 5.5
\subsection{Dependency Viewpoint}
\subsubsection{Design Concerns}
\subsubsection{Design Elements}
\paragraph{Dependency Attribute}
\subsubsection{Example Languages}

% 5.6
\subsection{Information Viewpoint}
\subsubsection{Design Concerns}
\subsubsection{Design Elements}
\paragraph{Dependency Attribute}
\subsubsection{Example Languages}

% 5.7
\subsection{Patterns Use Viewpoints}
\subsubsection{Design Concerns}
\subsubsection{Design Elements}
\subsubsection{Example Languages}

% 5.8
\subsection{Interface Viewpoints}
\subsubsection{Design Concerns}
\subsubsection{Design Elements}
\subsubsection{Example Languages}

% 5.9
\subsection{Structure Viewpoints}
\subsubsection{Design Concerns}
\subsubsection{Design Elements}
\subsubsection{Example Languages}

% 5.10
\subsection{Interaction Viewpoints}
\subsubsection{Design Concerns}
\subsubsection{Design Elements}
\subsubsection{Example Languages}

% 5.11
\subsection{State Dynamics Viewpoint}
\subsubsection{Design Concerns}
\subsubsection{Design Elements}
\subsubsection{Example Languages}

% 5.12
\subsection{Algorithms Viewpoints}
\subsubsection{Design Concerns}
\subsubsection{Design Elements}
\subsubsection{Process Attribute}
\subsubsection{Example}

% 5.13
\subsection{Resource Viewpoint}
\subsubsection{Design Concerns}
\subsubsection{Design Elements}
\paragraph{Resource Attribute}
\subsubsection{Examples}

% Annex A  Bibliography
% Annex B  Conforming design language description
% Annex C  Annex C Templates for an SDD

%\section{Conclusion}


%\pagebreak
%\bibliographystyle{IEEEtran}
%\bibliography{techreview}

\end{document}
