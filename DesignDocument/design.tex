\documentclass[letterpaper,10pt,titlepage,draftclsnofoot,onecolumn,onesided] {IEEEtran}
\usepackage{listings}
\usepackage{underscore}
\usepackage[bookmarks=true]{hyperref}
\usepackage[utf8]{inputenc}
\usepackage[english]{babel}
\usepackage{titling}
\usepackage{graphicx}
\usepackage[noadjust]{cite}
\nocite{*}
\usepackage{abstract}

% C: I added this package for the definitions portion of the document
\usepackage{amsthm}

\hypersetup{
    bookmarks=false,    % show bookmarks bar?
    pdftitle={Technology Review},    % title
    pdfauthor={Cramer Smith, Sam Lichlyter, Eric Winkler, Zach Schneider},                     % author
    pdfsubject={Design Document},                        % subject of the document
    pdfkeywords={IFT, Design, Postal}, % list of keywords
    colorlinks=true,       % false: boxed links; true: colored links
    linkcolor=black,       % color of internal links
    citecolor=black,       % color of links to bibliography
    filecolor=black,        % color of file links
    urlcolor=blue,        % color of external links
    linktoc=page            % only page is linked
} 

% Document Title:
\def\doctitle{A Tool to Automatically Organize the Structure of a Codebase Using Information Foraging Theory Design Patterns}
\def\doctype{Design Document}
\def\team{Team Postal | Group \#38}

\markboth{Oregon State University}{\doctitle}

\begin{document}

\title{\Huge{\bfseries{\textsf{\doctitle}}}\\\textsf{\Large{\doctype}}\\\textsf{\large{\team}}}
\author{Cramer Smith, Sam Lichlyter, Eric Winkler, Zach Schneider}

\maketitle
\vfill
%\begin{abstract}
%\end{abstract}
\vfill

\pagebreak

\tableofcontents

\pagebreak

% 1
\section{Overview}

% 1.1 
\subsection{Scope}
This document will cover the entirety design of the Postal extension written for the Visual Studio Code integrated development environment. 
The focus of the design will be on the four main parts of the extension, and the use of Information Foraging Theory within the extension.
The four parts of the extension design are the parser, the data structure, the interface with Visual Studio Code and the user interface.
The document with go through each of these parts and describe in detail how each will be implemented and how each part will function.
The Information Foraging Theory Patters that are planned to be explored within the extension are the Specification Matcher, Structural Relatedness, Impact Location, Path Search, and Recollection.
The document will go into more detail as to what these patterns mean and how they will influence the design of the extension.

% 1.2
\subsection{Purpose}
This design document describes the planned design and steps for implementing the Postal extension for Visual Studio Code. 
The team implementing the design will use this document as the blueprint for the implementation of the extension. 

% 1.3
\subsection{Intended Audience}
This document is meant for the design stakeholders. 
The design stake holders includes the team implementing the extension, their client, and their supervisors. 
The teams supervisors being the people grading the project on the implementation of the designs described within this document.

% 1.4 
\subsection{Conformance}
The document conforms to the IEEE Std 1016-2009.

% 2
\section{Definitions}
\newtheorem{VSC}{An acronym of Visual Studio Code. Visaul Studio Code is the IDE for which the postal extension is being built.}
\newtheorem{IDE}{An acronym of Integrated Development Environment.}
\newtheorem{UI}{An acronym of User Interface}
\newtheorem{MVC}{Model-View-Controller (MVC) design pattern assigns objects in an application one of three roles: model, view, or controller. The pattern defines not only the roles objects play in the application, it defines the way objects communicate with each other. Each of the three types of objects is separated from the others by abstract boundaries and communicates with objects of the other types across those boundaries. The collection of objects of a certain MVC type in an application is sometimes referred to as a layer—for example, model layer.\cite{appleMVC}}

% 3
\section{Conceptual Model for Software Design Descriptions}
The software will be loosely written with a model view control design pattern.
It is loosely MVC because it is an extension and some of the view will be out of the control of the extension, but the main parts of the extension will fill these roles .
The model will be the data structure that we will use to represent the parsed files. 
The view will be the user interface and the integrated development environment.
The IDE and the user interface that the extension creates will be the view and act dependent of each other.
The control will be the event handlers that are 

% 3.1
\subsection{Software Design in Context}
The extension is supposed to help novice web developer with organizing and create cleaner HTML and CSS code. 
To accomplish this the design of the extension it will contain a UI, a number of parsers for the different languages.
The parsers and the extensions will need to communicate between each other and VSC.
This communication will be facilitated by a datastructer that will also allow for a 



% 3.2
\subsection{Software Design Descriptions Within the Life Cycle}
\subsubsection{Influences on SDD Preparation}
\subsubsection{Influences on Software Life Cycle Products}
\subsubsection{Design Verification and Design Role in Validation}

% 4
\section{Design Description Information Content}

% 4.1
\subsection{Introduction}

% 4.2
\subsection{SDD identification}

% 4.3
\subsection{Design stakeholders and their concerns}

% 4.4
\subsection{Design views}

% 4.5
\subsection{Design viewpoints}

% 4.5
\subsection{Design elements}
\subsubsection{Design Entities}
\subsubsection{Design Attributes}
\paragraph{Name Attribute}
\paragraph{Type Attribute}
\paragraph{Purpose Attribute}
\paragraph{Author Attribute}
\subsubsection{Design Relationships}
\subsubsection{Design Constraints}

% 4.6
\subsection{Design Overlays}

% 4.7
\subsection{Design Rationale}

% 4.8
\subsection{Design Languages}

% 5
\section{Design Viewpoints}

% 5.1
\subsection{Introduction}

% 5.2
\subsection{Context Viewpoint}
\subsubsection{Design Concerns}
\subsubsection{Design Elements}
\subsubsection{Example Languages}

% 5.3
\subsection{Composition Viewpoint}
\subsubsection{Design Concerns}
\subsubsection{Design Elements}
\subsubsection{Function Attribute}
\subsubsection{Example Languages}

% 5.4
\subsection{Logical Viewpoint}
\subsubsection{Design Concerns}
\subsubsection{Design Elements}

% 5.5
\subsection{Dependency Viewpoint}
\subsubsection{Design Concerns}
\subsubsection{Design Elements}
\paragraph{Dependency Attribute}
\subsubsection{Example Languages}

% 5.6
\subsection{Information Viewpoint}
\subsubsection{Design Concerns}
\subsubsection{Design Elements}
\paragraph{Dependency Attribute}
\subsubsection{Example Languages}

% 5.7
\subsection{Patterns Use Viewpoints}
\subsubsection{Design Concerns}
\subsubsection{Design Elements}
\subsubsection{Example Languages}

% 5.8
\subsection{Interface Viewpoints}
\subsubsection{Design Concerns}
\subsubsection{Design Elements}
\subsubsection{Example Languages}

% 5.9
\subsection{Structure Viewpoints}
\subsubsection{Design Concerns}
\subsubsection{Design Elements}
\subsubsection{Example Languages}

% 5.10
\subsection{Interaction Viewpoints}
\subsubsection{Design Concerns}
\subsubsection{Design Elements}
\subsubsection{Example Languages}

% 5.11
\subsection{State Dynamics Viewpoint}
\subsubsection{Design Concerns}
\subsubsection{Design Elements}
\subsubsection{Example Languages}

% 5.12
\subsection{Algorithms Viewpoints}
\subsubsection{Design Concerns}
\subsubsection{Design Elements}
\subsubsection{Process Attribute}
\subsubsection{Example}

% 5.13
\subsection{Resource Viewpoint}
\subsubsection{Design Concerns}
\subsubsection{Design Elements}
\paragraph{Resource Attribute}
\subsubsection{Examples}

% Annex A  Bibliography
% Annex B  Conforming design language description
% Annex C  Annex C Templates for an SDD

%\section{Conclusion}


\pagebreak
\bibliographystyle{IEEEtran}
\bibliography{techreview}

\end{document}
