\documentclass[letterpaper,10pt,titlepage,draftclsnofoot,onecolumn,onesided] {IEEEtran}
\usepackage{listings}
\usepackage{underscore}
\usepackage[bookmarks=true]{hyperref}
\usepackage[utf8]{inputenc}
\usepackage[english]{babel}
\usepackage{titling}
\usepackage{graphicx}
\usepackage[noadjust]{cite}
\nocite{*}
\graphicspath{ {img/} }
\usepackage{abstract}

% C: I added this package for the definitions portion of the document
\usepackage{amsthm}

\newcommand{\namesigdate}[2][4cm]{%
  \begin{tabular}{@{}p{#1}@{}}
    #2 \\[2\normalbaselineskip] \hrule \\[0pt]
    {\small \textit{Signature}} \\[2\normalbaselineskip] \hrule \\[0pt]
    {\small \textit{Date}}
  \end{tabular}
}
\newcommand{\studentnamesigdate}[2][4cm]{%
  \begin{tabular}{@{}p{#1}@{}}
    #2 \\[2\normalbaselineskip] \hrule \\[0pt]
    {\small \textit{Signature}} \\[2\normalbaselineskip] \hrule \\[0pt]
    {\small \textit{Signature}} \\[2\normalbaselineskip] \hrule \\[0pt]
    {\small \textit{Signature}} \\[2\normalbaselineskip] \hrule \\[0pt]
    {\small \textit{Signature}} \\[2\normalbaselineskip] \hrule \\[0pt]
    {\small \textit{Date}}
  \end{tabular}
}

\hypersetup{
    bookmarks=false,    % show bookmarks bar?
    pdftitle={Progress Report},    % title
    pdfauthor={Cramer Smith, Sam Lichlyter, Eric Winkler, Zach Schneider},                     % author
    pdfsubject={Progress Report},                        % subject of the document
    pdfkeywords={IFT, Report, Postal}, % list of keywords
    colorlinks=true,       % false: boxed links; true: colored links
    linkcolor=black,       % color of internal links
    citecolor=black,       % color of links to bibliography
    filecolor=black,        % color of file links
    urlcolor=blue,        % color of external links
    linktoc=page            % only page is linked
} 

% Document Title:
\def\doctitle{A Tool to Automatically Organize the Structure of a Codebase Using Information Foraging Theory Design Patterns}
\def\doctype{Progress Report}
\def\team{Team Postal | Group \#38}

\markboth{Oregon State University}{\doctitle}

\begin{document}

\title{\Huge{\bfseries{\textsf{\doctitle}}}\\\textsf{\Large{\doctype}}\\\textsf{\large{\team}}}
\author{Cramer Smith, Sam Lichlyter, Eric Winkler, Zach Schneider}

\maketitle
\vfill

\vfill

\pagebreak

\tableofcontents

\pagebreak


\pagebreak

\section{Project Purpose and Goals}

\section{Current Status}
\subsection{Parsers}
The current implementation of the extension has one parser which parses HTML.
This parser is implemented in Perl using the Simple TokeParser which looks for specific tags and a specific attribute from that tag.
For example, this HTML parser specifically looks for the anchor (\verb|a|) tag and the hyperlink reference (\verb|href|) attribute and returns those values which are the links to either outside websites or internal pages.

\subsection{User Interface}
The user interface is currently running off of dummy, static data. 
It opens up in a new window using a program called Electron as opposed to a web browser to cut down on the overhead associated with all the complexities of the web browser.
It shows each web page as a node and each link from one page to another as an edge in a graph. See Figure 1.

\subsection{Data Layer}


\section{Problems and Solutions}
	\subsection{Pre-Week 3}

	\subsection{Week 3}

	\subsection{Week 4}

	\subsection{Week 5}

	\subsection{Week 6}

	\subsection{Week 7}

	\subsection{Week 8}

	\subsection{Week 9}

	\subsection{Week 10}

\section{Prototype Code Samples}
	
	\subsection{FileMap UI}
	\begin{lstlisting}

	\end{lstlisting}

	\subsection{HTML Parser}
	\begin{lstlisting}

	\end{lstlisting}

\section{Fall Term Retrospective}
	\begin{center}
		\begin{tabular}{ |  p{0.25\linewidth}  |  p{0.25\linewidth}  | p{0.25\linewidth} | p{0.25\linewidth} |}
		\hline
		Topic & Positives & Deltas & Actions \\ \hline
		Communication with client & 
		Flexible guy & 
		Frequency and clarity of communication need to be improved &
		We will meet and email about progress more \\ \hline

		\hline
		\end{tabular}
	\end{center}


\pagebreak
\bibliographystyle{IEEEtran}
\bibliography{progress}
\pagebreak

\namesigdate{Client Signature} \hfill 
\studentnamesigdate[4cm]{Student Signatures}
\end{document}
