
\documentclass[letterpaper,10pt,titlepage,draftclsnofoot,onecolumn] {IEEEtran}

\usepackage{geometry}
\geometry{textheight=8.5in, textwidth=6in}

\newcommand{\cred}[1]{{\color{red}#1}}
\newcommand{\cblue}[1]{{\color{blue}#1}}

\newcommand{\toc}{\tableofcontents}

\def\name{Cramer Smith, Sam Lichlyter, Eric Winkler and Zach Schneider}

\usepackage{titling}
\title{Requirements}
\author{Cramer Smith, Sam Lichlyter, Eric Winkler and Zach Schneider}
\date{October 28th, 2016}

\parindent = 0.0 in
\parskip = 0.1 in

\begin{document}

\begin{titlepage}

\maketitle
\begin{center}
CS461: CS Senior Capstone \\
Fall 2016

\begin{abstract}

A requirements document explains why a product is needed, puts the product in context, and describes what the finished product will be like. A large part of the requirements document is the formal list of requirements.

\end{abstract}

\end{center}

\end{titlepage}

\section{Requirements}

The Visual code that we 



\end{document}

